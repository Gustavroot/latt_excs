\documentclass[11pt]{article}
%Gummi|065|=)
\title{\textbf{Selected Topics of LGT}}
\author{Gustavo Ramirez\\
		Assignment \# 4}
\date{}
\begin{document}

\maketitle

Incomplete...


\section{a), b), c), d), e)}

The idea is that, when comparing against the continuum dispersion relation, the lattice dispersion relation doesn't seem to approach to the continuum one. But this is only an artifact of the way we have defned our simulation variables. If we scale our momentum, energy and mass with $a$, and plot using physical values only (i.e. $p\rightarrow ap$, $E\rightarrow aE$, $m\rightarrow am$), then we will see how we are actually approaching the continuum limit. There will remain, though, gaps for high momenta regions.

To know which value of $a$ to use, note that, in the previous exercise, we have taken: $a=1, \ m=0.4$. But, what we have actually done there is to take $ma=0.4$. Then, when switching for example to $m=0.2$, we are changing to $ma=0.2=(0.4)a \rightarrow a = 0.5$, and therefore it makes sense to double both space and time values of $N$, to compare again against the same volume.


\end{document}
